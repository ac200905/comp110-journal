% Please do not change the document class
\documentclass{scrartcl}

% Please do not change these packages
\usepackage[hidelinks]{hyperref}
\usepackage[none]{hyphenat}
\usepackage{setspace}
\doublespace




% You may add additional packages here
\usepackage{amsmath}
\usepackage{cite}
\setlength{\parindent}{4em}
\usepackage{graphicx}

% Please include a clear, concise, and descriptive title
\title{Computing Machinery and Intelligence
}

% Please do not change the subtitle
\subtitle{COMP110 - Research Journal}

% Please put your student number in the author field
\author{1703170}

\begin{document}

\maketitle

\abstract{abstract here... }

\section{Learning Machines}


When Turing wrote "Computing Machinery and Intelligence" \cite{turing1950computing:1}, no Artificial Intelligence programs existed, indeed the phrase Artificial Intelligence wasn't used until 1956, just 2 years after his untimely death. General purpose computers were rare enough that the article was mostly theoretical in nature. \par
Within a few years of the article, more computers were being created and the musings of Turing were finally able to be put to the test, albeit with limitations. Turing's idea of Learning Machines for example \cite{ramscar2010computing:2}. Turing argued that a Learning Machine was the likely the only realistic way of producing human-like intelligence in computers, hand coding every conceivable eventuality in advance to simulate a human brain was unlikely to be possible. \par
In Daniel Dennett's book, Darwin's Dangerous Idea: Evolution and the Meanings of Life \cite{dennett1996darwin:3}ch8sec5, Dennett references Turing's "prophetic essay" and goes on to talk about Arthur Samuel's experiments in creating perhaps one of the first candidates that could be called Artificial Intelligence \cite{samuel2000some:4}. In Samuels paper, he describes the learning schemes he was able to program into the computer to play checkers and learn from previous games with itself to improve. Samuels learning version of the checkers program was finished in 1955, just 5 years after Turing's paper. Although by no means perfect, the program was sufficient enough that it surpassed Samuel's proficiency at checkers. It may be less complicated a game compared with chess, nonetheless, this shows the foresight of Turing when in his paper he concludes by suggesting that machines should be taught abstract tasks, such as chess, in order to eventually progress to a time when machines can compete with humans on an intellectual level. \par
Dennett also notes in his book that this ability to learn is classically Darwinian \cite{darwin2009origin:5}. One might infer that Turing's paper can be interpreted as a prelude to the idea of Artificial Evolution.




\section{The Imitation Game}

Arguably the most influential aspect of the paper, widely viewed as it's most controversial, is the Imitation Game. Now known as the Turing test \cite{suchman1987plans:6}. Initially a parlour game in which an interrogator attempted to distinguish a man from a woman using only the worded responses they both returned from questions put to them by the interrogator. Turing re-purposed this to accept a man and a machine instead. The theory being, that if the interrogator was unable to distinguish the man from the machine then this would be sufficient evidence for machine intelligence. \par
\cite{weizenbaum1966eliza:7}

\section{Section 2}



\section{Section 3}
Noam Ch


\section{Key Contributions}
Philosophy of the mind.

Imitation game
Descartes
Eliza programs, joseph Weizenbaum

Learning Machines

“Instead of trying to produce a programme to simulate the adult mind, why not rather try to produce one which simulates the child’s?”
Suggests chess, Arthur Samuel - checkers


\section{Conclusion}



\bibliographystyle{ieeetran}
\bibliography{references}

\end{document}




