% Please do not change the document class
\documentclass{scrartcl}

% Please do not change these packages
\usepackage[hidelinks]{hyperref}
\usepackage[none]{hyphenat}
\usepackage{setspace}
\doublespace




% You may add additional packages here
\usepackage{amsmath}
\usepackage{cite}
\setlength{\parindent}{4em}
\usepackage{graphicx}

% Please include a clear, concise, and descriptive title
\title{Review: Computing Machinery and Intelligence
}

% Please do not change the subtitle
\subtitle{COMP110 - Research Journal}

% Please put your student number in the author field
\author{1703170}

\begin{document}

\maketitle

\section{The Imitation Game}

\subsection{The DOCTOR}

Arguably the most influential aspect of Alan M Turing's revolutionary paper \textquotedblleft Computing Machinery and Intelligence" \cite{turing1950computing:1}, and widely viewed as it's most controversial, is the Imitation Game. Influenced perhaps by Descartes \cite{descartes1996discourse:2}, now known as the Turing test \cite[p.17]{suchman1987plans:3}. Initially a parlour game in which an interrogator attempted to distinguish a man from a woman using only the worded responses they both returned from questions put to them by the interrogator. Turing re-purposed this to accept a man and a machine instead. The theory being, that if the interrogator was unable to distinguish the man from the machine then this would be sufficient evidence for machine intelligence.

A simple form of the Turing test was said to have been beaten using a computer program in 1966 called ELIZA \cite{weizenbaum1966eliza:4}, using a version of the program called DOCTOR. The program essentially accepted input data from the user in text form and output responses related to certain key words from the perspective of a therapist. This helped the machine assume a sense of distance from the user and allowed for vague responses, which in turn made the realization that it was a machine more unlikely.

This clearly doesn't pass the Turing test if any amount of rigour is injected into the interrogation. While acknowledging that some scientists disagree with the usefulness of the Turing test in modern AI science \cite{hayes1995turing:5}, it is nevertheless a large leap towards the possibility of a machine one day passing the test. What once was merely theoretical, suddenly seems within the bounds of reality. Clearly, this shows how ahead of his time Turing was.

\textit{\textquotedblleft We can only see a short distance ahead, but we can see plenty there that needs to be done."} 

\subsection{CAPTCHA Science}

\textit{\textquotedblleft Completely Automated Public Turing Test to Tell Computers and Humans Apart."}

One of the influences that Turing's paper had, was on the invention of CAPTCHA. Essentially a Turing test like the Imitation Game proposal, except the human interrogator is now replaced by a computer\cite{von2004telling:6}. With CAPTCHA, the idea is to create a test that humans can pass most of the time but a computer cannot. The most common versions utilize visual tests \cite{fischer2006visual:7}, computers find processing visual data a lot harder than humans intuitively do. This type of test allows CAPTCHA to be useful in many modern applications especially in the field of online security. 

The internet has revolutionised the way information is spread and gathered. Websites often hold online polls for example. In order to prevent users from voting more than once, the IP address is noted by the website, however programs can be created to circumvent the IP problem and to vote multiple times with fake IP addresses. If online polling systems can be taken advantage of so easily, how can they be trusted?  This is where CAPTCHAs come into the fold. If a CAPTCHA system is in play, it forces the user to prove it's human, thus rendering the bot program useless. CAPTCHAs can be used in the same fashion to prevent spammers from creating bots to generate spam email accounts or to help ticket brokers prevent the wholesale buying of tickets to events.

These tests aren't always perfect however and new algorithms are continually developed to improve them \cite{mori2003recognizing:8}. When CAPTCHAs are beaten by programs, it can be seen in a positive light in that this is further evidence of improving Artificial Intelligence.

It is interesting how a test once conceived of in an effort to one day prove that machines could think is now being utilized with the expectation that they could not, or at least cannot currently think. 




\section{Learning Machines}

\subsection{Of Checkers and Chess}

When Turing wrote "Computing Machinery and Intelligence" \cite{turing1950computing:1}, no Artificial Intelligence programs existed, indeed the phrase Artificial Intelligence wasn't used until 1956, just 2 years after his untimely death. General purpose computers were rare enough that the article was mostly theoretical in nature. \par
Within a few years of the article, more computers were being created and the musings of Turing were finally able to be put to the test, albeit with limitations. Turing's idea of Learning Machines for example \cite{ramscar2010computing:9}. Turing argued that a Learning Machine was likely the only realistic way of producing human-like intelligence in computers, hand coding every conceivable eventuality in advance to simulate a human brain was unlikely to be possible. 

In Daniel Dennett's book, Darwin's Dangerous Idea: Evolution and the Meanings of Life \cite[p.207-212]{dennett1996darwin:10}, Dennett references Turing's \textit{\textquotedblleft prophetic essay"} and goes on to talk about Arthur Samuel's experiments in creating perhaps one of the first candidates that could be called Artificial Intelligence \cite{samuel2000some:11}. In Samuel's paper, he describes the learning schemes he was able to program into the computer to play checkers and learn from previous games with itself to improve. Samuel's learning version of the checkers program was finished in 1955, just 5 years after Turing's paper. Although by no means perfect, the program was sufficient enough that it surpassed Samuel's proficiency at checkers. It may be less complicated a game compared with chess, nonetheless, this shows the foresight of Turing when in his paper he concludes by suggesting that machines should be taught abstract tasks, such as chess, in order to eventually progress to a time when machines can compete with humans on an intellectual level \cite[p.26-30]{dawkins2016selfish:12}. 

Dennett also notes in his book that this ability to learn is classically Darwinian \cite{darwin2009origin:13}. One might infer that Turing's paper can be interpreted as a prelude to the idea of Artificial Evolution not just Artificial Intelligence. The comparisons are there, perhaps one day we might write programs that machines might use to evolve beyond them. We humans after all have our own natural base code in the form of DNA \cite[p.66-68]{dawkins2016selfish:12}. Is it such a leap to conclude that one day machines might learn to learn more and more as all natural life on earth does? The only real challenge is the immense complexity required of that initial code.

\subsection{Philosophy}

The philosophical implications spawned out of the paper's publication can be seen throughout our culture. Not only in the sciences but also pop culture in general. From classic films and books like \textit{I,Robot} (published in the same year as Turing's paper), \textit{2001 :A Space Odyssey} and \textit{Blade Runner} to the more recent in \textit{Chappie} and \textit{Ex Machina}. This popularity emphasizes the notion that we all cannot help but be intrigued by the question Turing poses: \textit{\textquotedblleft Can machines think?"}. 

Would machines with AI even have the negative characteristics that we see in humans? When Turing talks about a machine being able to imitate a human, he may be giving humans too much credit. It may be better to think of a machine not imitating, but superseding humans. This depends on how important we believe emotions are to the idea of being intelligent and how we see general intelligence \cite{mccarthy1969some:14},\cite{brooks1999cambrian:15}. An argument could be made that machines could one day be superior to humans in every way but in emotional intelligence. Would this be enough to consider them alive \cite[p.21-32]{picard1997affective:16}? 

The paper also gives rise to a number of moral implications surrounding the development of AI. As AI becomes more readily available to people, we must start to consider who is liable or responsible in certain scenarios \cite[p84-91]{kaplan2015humans:17}. When before machines were just tools, soon they may become aware enough to act on their own. Parent's are responsible for their child's actions until the day the child has grown enough to become responsible for her own actions. At what stage does the AI become responsible for it's actions over the one who programmed it?

For example, if thinking is just a kind of computation \cite{pinker2003mind:18} and machines do one day become self-aware and have little else other than their make-up to distinguish them from Humans (eg. genetic vs mechanical), on what basis can we consider humans special?

\section{Conclusion}
These are just some of the influences and contributions that the paper \textquotedblleft Computing Machinery and Intelligence" has made since it's publication in 1950. With hindsight, this paper could be considered the birth of Artificial Intelligence as we now know it and has sparked a plethora of philosophical debate ever since. \par
In his paper, Turing wrongly predicted that by the year 2000 computers would be powerful enough that machines would be thinking and winning at the Imitation Game with a 70 percent success rate. He may have been wrong on the date, but with continued evolution in the technical sciences, including Quantum Computing, his prediction may still one day come true. A more interesting question might then be; machines can think, but can they feel?







\bibliographystyle{ieeetran}
\bibliography{references}

\end{document}




